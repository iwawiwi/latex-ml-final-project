%%%%%%%%%%%%%%%%%%%%%%%%%%%%%%%%%%%%%%%%%%%%%%%%%%%%%%%%%%%%%%%%%%
%%%%%%%% ICML 2014 EXAMPLE LATEX SUBMISSION FILE %%%%%%%%%%%%%%%%%
%%%%%%%%%%%%%%%%%%%%%%%%%%%%%%%%%%%%%%%%%%%%%%%%%%%%%%%%%%%%%%%%%%

% Use the following line _only_ if you're still using LaTeX 2.09.
%\documentstyle[icml2014,epsf,natbib]{article}
% If you rely on Latex2e packages, like most moden people use this:
\documentclass{article}

% use Times
\usepackage{times}
% For figures
\usepackage{graphicx} % more modern
%\usepackage{epsfig} % less modern
\usepackage{subfigure} 
%\usepackage{subcaption}
\usepackage{multirow}
\usepackage{booktabs}
\usepackage{url}
\usepackage{amssymb}

% For citations
\usepackage{natbib}

% For algorithms
\usepackage{algorithm}
\usepackage{algorithmic}

% As of 2011, we use the hyperref package to produce hyperlinks in the
% resulting PDF.  If this breaks your system, please commend out the
% following usepackage line and replace \usepackage{icml2014} with
% \usepackage[nohyperref]{icml2014} above.
\usepackage{hyperref}

% Packages hyperref and algorithmic misbehave sometimes.  We can fix
% this with the following command.
\newcommand{\theHalgorithm}{\arabic{algorithm}}

% Employ the following version of the ``usepackage'' statement for
% submitting the draft version of the paper for review.  This will set
% the note in the first column to ``Under review.  Do not distribute.''
\usepackage{icml2014} 
% Employ this version of the ``usepackage'' statement after the paper has
% been accepted, when creating the final version.  This will set the
% note in the first column to ``Proceedings of the...''
%\usepackage[accepted]{icml2014}


% The \icmltitle you define below is probably too long as a header.
% Therefore, a short form for the running title is supplied here:
\icmltitlerunning{Submission and Formatting Instructions for ICML 2014}

\begin{document} 

\twocolumn[
\icmltitle{Multi-label Classification using Ensemble-based Extreme Machine Learning}

% It is OKAY to include author information, even for blind
% submissions: the style file will automatically remove it for you
% unless you've provided the [accepted] option to the icml2014
% package.
\icmlauthor{I Wayan Wiprayoga Wisesa}{i.wayan31@ui.ac.id}
\icmladdress{Faculty of Computer Science Universitas Indonesia,
            Depok, West Java, Indonesia}
\icmlauthor{M. Ivan Fanany}{fanany@cs.ui.ac.id}
\icmladdress{Faculty of Computer Science Universitas Indonesia,
            Depok, West Java, Indonesia}

% You may provide any keywords that you 
% find helpful for describing your paper; these are used to populate 
% the "keywords" metadata in the PDF but will not be shown in the document
\icmlkeywords{boring formatting information, machine learning, ICML}

\vskip 0.3in
]

\begin{abstract} 
%There are two review cycles for ICML 2014, with full paper submissions due on October 4, 2013 and January 31, 2014. Reviewing will be blind to the identities of the authors, and therefore identifying information must not appear in any way in papers submitted for review. Submissions must be in PDF, with an 8 page length limit ( 9 pages are allowed if the 9-th page contains \emph{only} references).

Lorem ipsum dolor sit amet, aliquip molestiae sit id. Pro cu etiam propriae constituto. Persecuti cotidieque ut duo, id veritus accommodare pro. Accusam erroribus in eam, te usu porro delectus reformidans, magna oratio mucius no mei. Duo in audiam principes laboramus, in mel reque pertinacia contentiones, eos cetero luptatum postulant at. Vix everti corpora insolens in. Eu vim facer decore. Nec ei duis eloquentiam, habeo iuvaret complectitur mel te. Eu ius labores propriae, laudem equidem no sea. Veritus delicatissimi eu mel. Ut enim decore eruditi has, eam id harum convenire sententiae.

\end{abstract} 


\section{Introduction}
\label{intro}

%Klasifikasi dapat dipandang sebagai masalah pemetaan suatu fungsi $f\colon x \mapsto y$, dengan $x$ merupakan atribut atau fitur data dan $y$ merupakan kelas target yang bersesuaian dengan $x$. Pada umumnya $x$ dapat memiliki dimensi yang tinggi dan $y$ berdimensi 1. Jika $y \in \{0,1\}$, maka disebut problem \textit{binary classification}$. Jika $y \in \{1,2,\dots,N}$, maka disebut \textit{multiclass classification}. Dilain sisi, pada masalah klasifikasi multi-label, $y$ dapat memiliki dimensi yang tinggi.

Classification can be seen as the function mapping problem $f\colon x \mapsto y$, with $x$ denoted the attributes or features of the data and $y$ denoted the target class with respect to input $x$. In common case, $x$ can have a high dimensional size and $y$ is one dimensional. If $y \in \{0,1\}$, it is called a \textit{binary classification} problem. In other word, the binary classification dealt with problem that only have two possible target value: positive (true) or negative (false). If $y \in \{L_1,L_2,\dots,L_N\}$ and $|N| > 1$, the problem is called a \textit{multi-class classification} problem. In multi-class classification problem, there are many (more than one) possible target value. On the other side, in multi-label classification, the target $y \subseteq \{L_1,L_2\dots,L_N\}$, where $|N| > 1$. In other word, the dimension of target $y$ can also have one or a high dimensional size because the possible value of the target is the subset of all possible label.

Multi-label classification are increasingly required by current modern application. Many real world problem can be categorized as multi-label classification problem. Music categorization is one example of such problem. In the problem of music categorization, a song or music can be categorized into one or more music genre. For example, the song titled \textit{100\%}--performed by Mariah Carey, can be categorized into \textit{R\&B}, \textit{soul} and \textit{gospel} song category \footnote{\url{http://en.wikipedia.org/wiki/100\%25\_\%28Mariah\_Carey\_song\%29}}. Some other application of multi-label classification including text categorization, functional genomics, and scene classification.

Multi-label classification problem is more difficult to solve than single-label one, either binary or multi-class classification \cite{Zhang2007}. Several literature have been conducted an overview on some approaches in order to dealt with multi-label learning problem such as work by \cite{Tsoumakas2007} and \cite{Madjarov2012}. In their works, \cite{Tsoumakas2007} have been summarized some of approaches in the multi-label classification problem and categorized them into two categories, (a) problem transformation method and (b) algorithm adaptation method. Later, \cite{Madjarov2012} extend these categories by including (c) ensemble method approach.

In the problem transform approach, the multi-label classification problem are transformed into one or more single-label (either binary or multi-class classification) problem. Then, from this transformed problem, a common single-label learning algorithm is used to solve the problem. There are several technique of problem transform approach such as, binary relevance, label power-set, and pair-wise method \cite{Madjarov2012}. In other hand, the algorithm adaptation method extend and customize an existing machine learning algorithm in order to perform the task of multi-label classification. The ensemble method category then are developed based on the ensemble learning framework and using combination of top of the problem transformation and algorithm adaptation method \cite{Madjarov2012}.

In this paper, the experiment on multi-label learning system which is based on the combination of ensemble learning and problem transformation approaches has been conducted. A new voting scheme for the ensemble system hypotheses is proposed in order to solve the multi-label classification problem. The system have been tested by using several publicly dataset on multi-label classification.

\section{Previous Work}
\label{related}

Multi-label classification problem has attracted several researcher for over past  ten years. \cite{Zhang2007} have been proposed a multi-label classification technique based on infamous $k$NN method named ML-$k$NN....


\section{Preliminaries}
\label{preliminary}

\subsection{Extreme Learning Machine}
\label{elm}

Extreme Learning Machine (ELM) is one of machine learning algorithm that developed by \cite{Huang2004}. ELM is the generalization of single-hidden layer feedforward neural network (SLFN) which its weights are initialized randomly and then the output weight are computed through inverse operation of hidden layer output matrices. ELM have the advantage in the learning phase computational time. \cite{Huang2004} claimed that, ELM can run the training process thousand time faster than traditional feedforward network learning algorithm and also have better generalization performance. The following steps are the general training step of Extreme Learning Machine.
\begin{itemize}
\item \textbf{step 1}: randomly assign input weight $w_i$ and bias $b_i$ for $i = {1,2,\dots,N}$ 
\item \textbf{step 2}: calculate the hidden layer output matrices $H$ based on any activation function, input weight, and bias. 
\item \textbf{step 3}: calculate the output weight $\beta$ by using inverse of hidden layer output matrices.
\end{itemize}


\subsection{Ensemble Framework}
\label{ensemble}

An ensemble framework is one of popular supervised learning method in machine learning. It combines several diverse model to obtain a good classification performance. The idea behind the ensemble framework is to combine several independent classifier algorithm that can produce hypotheses better than random guessing. Such classifier algorithm is called weak learner in the ensemble system. Final hypotheses of the ensemble system then concluded based on the voting.

There are 4 important component in the ensemble framework system \cite{Rokach2010}, there are:

\begin{itemize}
\item Training set--A labeled dataset used for training the ensemble system.
\item Base inducer--or sometimes called weak learner in the ensemble system. It is an induction algorithm that form hypotheses or generalized relationship between input attributes and the target (label).
\item Diversity generator--The component that responsible in generating independence and diversity of inducer hypotheses.
\item Combiner--The component that responsible to combine all hypotheses made by all the inducer hypotheses in the ensemble system.
\end{itemize}

Through the ensemble system, we can form a final hypotheses whose performance is better than all the individual weak classifier in the system \cite{Rokach2010}. This is based on the well-known Condrocet's jury theorem, "If each voter has a probability $p$ of being correct and the probability of a majority of voters being correct is $L$, then $p > 0.5$ implies $L > p$ and also $L$ approaches $1$, for all $p > 0.5$ as the number of voters approcahes infinity". There are several famous method in ensemble system, it is include

%Eu vim facer decore. Nec ei duis eloquentiam, habeo iuvaret complectitur mel te. Eu ius labores propriae, laudem equidem no sea. Veritus delicatissimi eu mel. Ut enim decore eruditi has, eam id harum convenire sententiae.


\section{System Overview}
\label{overview}

Before the data is trained, problem transformation approach are applied on the data. For example, if we have two data example $A$ and $B$, $A$ is labeled with $\{1,2\}$ and $B$ is labeled with $\{1,3,4\}$, the transformed data will consist total 5 data. The illustration of the example of this transformed data can be seen on Table \ref{tab:transformation}.

\begin{table}[t]
\caption{The example of problem transformation approach in the experiment}
\label{tab:transformation}
\vskip 0.15in
\begin{center}
\begin{small}
\begin{sc}
\begin{tabular}{cc}
\hline
\abovespace\belowspace
Instance & Label \\
\hline
\abovespace
$A$    	& 1  \\
$A$ 	& 2 \\
$B$ 	& 1 \\
$B$ 	& 3 \\
\belowspace
$B$ 	& 4 \\
\hline
\end{tabular}
\end{sc}
\end{small}
\end{center}
\vskip -0.1in
\end{table}

Then from these The proposed ensemble system is using bagging method to build each weak learner model. 

%Minim dolorem convenire sit ne, veniam everti labores mel ut, vim aperiam laboramus id. Lorem tractatos sed ea \cite{Zhang2007}. Mucius verterem te qui, sed malorum maiestatis dissentias ei \cite{Read2011}. Probo paulo doming ne vix, ius oratio sententiae in. In quem vero velit vis, mea accusata scribentur vituperatoribus ei. Aperiam mnesarchum nec at, vel ea aliquip ornatus. Mea efficiendi definitionem et, tritani erroribus sit et \cite{Trohidis2008}.

\begin{algorithm}[tb]
   \caption{Proposed system's learning and testing algorithm}
   \label{alg:example}
\begin{algorithmic}
   \STATE {\bfseries Input:} (1) multi-labeled data $\{(x_1, y_1),\dots,(x_n,y_n)\}$, in the form of $X \in \mathbb{R}^{n \times k}$ and $Y \in \mathbb{Z}^{n \times l}$; (2) weak learner $H$.
   \STATE \textbf{step 1:} Transform each example in dataset.
   \STATE \textbf{step 2:} Create $M$ weak learner $H$.
   \STATE \textbf{step 3:} Train each weak learner $H$ using bagging (randomly choose the data with replacement) strategy from transformed dataset.
   \STATE \textbf{step 4:} For the testing phase, from test data, create hypotheses by using each weak learner $H$.
   \STATE \textbf{step 5:} each hypotheses of $H$ are counted, overall hypotheses of ensemble system are done by applying some threshold value.
\end{algorithmic}
\end{algorithm}


\section{Experiment and Discussion}
\label{expr}

\subsection{Experiment on Public Datasets}
Several public datasets available on the Internet have been used as the experiment for ensemble system that have been developed. There are 3 different datasets for multi-label classification that are used on the experiment. These dataset is publicly available on MULAN website\footnote{\url{http://mulan.sourceforge.net/datasets.html}}. Table \ref{tab:dataset} show the overview of the dataset used in the experiment.

\begin{table}[t]
\caption{Comparison overview of several multi label dataset used in experiment.}
\label{tab:dataset}
\vskip 0.15in
\begin{center}
\begin{small}
\begin{sc}
\begin{tabular}{lcccr}
\hline
\abovespace\belowspace
Dataset & Instances & Labels & Cardinality \\
\hline
\abovespace
Emotion    	& 593 & 6 & 1.869 \\
Scene 		& 2407 & 6 & 1.074 \\
\belowspace
Yeast    	& 2417 & 14 & 4.237 \\
\hline
\end{tabular}
\end{sc}
\end{small}
\end{center}
\vskip -0.1in
\end{table}

\begin{figure*}[ht]
\vskip 0.2in
	\begin{subfigure}{}
	\centering
		\includegraphics[width=2.1in]{assets/[EMOTION][1]accuracy_profile_@10-fold_cv.png}
		\includegraphics[width=2.1in]{assets/[EMOTION][1]precision_profile_@10-fold_cv.png}
		\includegraphics[width=2.1in]{assets/[EMOTION][1]recall_profile_@10-fold_cv.png}
		\includegraphics[width=2.1in]{assets/[SCENE][1]accuracy_profile_@10-fold_cv.png}
		\includegraphics[width=2.1in]{assets/[SCENE][1]precision_profile_@10-fold_cv.png}
		\includegraphics[width=2.1in]{assets/[SCENE][1]recall_profile_@10-fold_cv.png}
		\includegraphics[width=2.1in]{assets/[YEAST][1]accuracy_profile_@10-fold_cv.png}
		\includegraphics[width=2.1in]{assets/[YEAST][1]precision_profile_@10-fold_cv.png}
		\includegraphics[width=2.1in]{assets/[YEAST][1]recall_profile_@10-fold_cv.png}
		\caption{Accuracy (left column), precision (midlle column), and recall (right column) profile of bagging ELM, bagging SVM, and bagging decision stump for dataset \texttt{EMOTION} (upper row), \texttt{SCENE} (middle row), and \texttt{YEAST} (bottom row). System evaluated by using 10-fold cross validation scheme.}
	\end{subfigure}
\vskip -0.2in
\end{figure*} 

\subsection{Experiment on Selection of Ensemble Decision}
Exerci reformidans nam no, has et error impedit omittantur. Ei hinc senserit qui. Ea qui duis justo primis, vide consul eam id, odio justo detraxit no eos. Fastidii gloriatur vis id. Pri noster reprimique vituperatoribus ea, vis ut altera gloriatur. Viderer omittam indoctum in pro.


\section{Conclusions}
\label{conculsion}

Ex corpora platonem omittantur duo, pri semper efficiantur an. Adipisci constituam eam cu, graece legendos nominati ad sed, pri ea suas delicata. Purto oratio in quo, vis ocurreret forensibus at. Ut placerat definiebas est, vix no sumo epicurei electram, postea regione blandit qui at. Labitur intellegebat voluptatibus ius ea. Ex has purto solum, eum phaedrum efficiantur ut. Et nec facer soluta cetero, ad integre rationibus est.

%\section{Electronic Submission}
%\label{submission}

%As in the past few years, ICML will rely exclusively on electronic formats for submission and review. 


%\subsection{Templates for Papers}

%Electronic templates for producing papers for submission are available for \LaTeX\/ . Templates are accessible on the World Wide Web at:\\ \textbf{\texttt{http://icml.cc/2014/}}

%\noindent
%Send questions about these electronic templates to \texttt{program@icml.cc}.

%The formatting instructions below will be enforced for initial submissions and camera-ready copies. 
%\begin{itemize}
%\item The maximum paper length is 8 pages excluding references, and 9 pages including references.
%\item Do not alter the style template; in particular, do not compress the paper format by reducing the vertical spaces.
%\item Do not include author information or acknowledgments in your initial submission. 
%\item Place figure captions {\em under} the figure (and omit titles from inside the graphic file itself).  Place table captions {\em over} the table.
%\item References must include page numbers whenever possible and be as complete as possible.  Place multiple citations in chronological order.  
%\end{itemize}
%Please see below for details on each of these items.

%\subsection{Submitting Papers}

%Submission to ICML 2014 will be entirely electronic, via a web site (not email).  The URL and information about the submission process are available on the conference web site at

%\textbf{\texttt{http://icml.cc/2014/}}

%{\bf Paper Deadline:} The deadline for paper submission to ICML 2014 is at 23:59 Universal Time (3:59 Pacific Daylight Time) on the due dates (October 4, or January 31, depending on the review cycle). If your full submission does not reach us by this time, it will not be considered for publication. There is no separate abstract submission.

%{\bf Anonymous Submission:} To facilitate blind review, no identifying author information should appear on the title page or in the paper itself.  Section~\ref{author info} will explain the details of how to format this.

%{\bf Simultaneous Submission:} ICML will not accept any paper which, at the time of submission, is under review for another conference or has already been published. This policy also applies to papers that overlap substantially in technical content with conference papers under review or previously published. ICML submissions must not be submitted to other conferences during ICML's review period. Authors may submit to ICML substantially different versions of journal papers that are currently under review by the journal, but not yet accepted at the time of submission. Informal publications, such as technical reports or papers in workshop proceedings which do not appear in print, do not fall under these restrictions.

%\medskip

%To ensure our ability to print submissions, authors must provide their manuscripts in \textbf{PDF} format.  Furthermore, please make sure that files contain only Type-1 fonts (e.g.,~using the program {\tt pdffonts} in linux or using File/DocumentProperties/Fonts in Acrobat).  Other fonts (like Type-3) might come from graphics files imported into the document.

%Authors using \textbf{Word} must convert their document to PDF. Most of the latest versions of Word have the facility to do this automatically.  Submissions will not be accepted in Word format or any format other than PDF. Really. We're not joking. Don't send Word.

%Those who use \textbf{\LaTeX} to format their accepted papers need to pay close attention to the typefaces used.  Specifically, when producing the PDF by first converting the dvi output of \LaTeX\ to Postscript the default behavior is to use non-scalable Type-3 PostScript bitmap fonts to represent the standard \LaTeX\ fonts. The resulting document is difficult to read in electronic form; the type appears fuzzy. To avoid this problem, dvips must be instructed to use an alternative font map.  This can be achieved with something like the following commands:\\[0.5em] {\bf dvips -Ppdf -tletter -G0 -o paper.ps paper.dvi}\\ {\bf ps2pdf paper.ps}\\[0.5em] Note that it is a zero following the ``-G''.  This tells dvips to use the config.pdf file (and this file refers to a better font mapping).

%Another alternative is to use the \textbf{pdflatex} program instead of straight \LaTeX. This program avoids the Type-3 font problem, however you must ensure that all of the fonts are embedded (use {\tt pdffonts}). If they are not, you need to configure pdflatex to use a font map file that specifies that the fonts be embedded. Also you should ensure that images are not downsampled or otherwise compressed in a lossy way.

%Note that the 2014 style files use the {\tt hyperref} package to make clickable links in documents.  If this causes problems for you, add {\tt nohyperref} as one of the options to the {\tt icml2014} usepackage statement.

%\subsection{Reacting to Reviews}
%We will continue the ICML tradition in which the authors are given the option of providing a short reaction to the initial reviews. These reactions will be taken into account in the discussion among the reviewers and area chairs.

%\subsection{Submitting Final Camera-Ready Copy}

%The final versions of papers accepted for publication should follow the same format and naming convention as initial submissions, except of course that the normal author information (names and affiliations) should be given.  See Section~\ref{final author} for details of how to format this.

%The footnote, ``Preliminary work.  Under review by the International Conference on Machine Learning (ICML).  Do not distribute.'' must be modified to ``\textit{Proceedings of the $\mathit{31}^{st}$ International Conference on Machine Learning}, Beijing, China, 2014.  JMLR: W\&CP volume 32. Copyright 2014 by the author(s).''

%For those using the \textbf{\LaTeX} style file, simply change $\mathtt{\backslash usepackage\{icml2014\}}$ to 

%\verb|\usepackage[accepted]{icml2014}|

%\noindent
%Authors using \textbf{Word} must edit the footnote on the first page of the document themselves.

%Camera-ready copies should have the title of the paper as running head on each page except the first one.  The running title consists of a single line centered above a horizontal rule which is $1$ point thick. The running head should be centered, bold and in $9$ point type.  The rule should be $10$ points above the main text.  For those using the \textbf{\LaTeX} style file, the original title is automatically set as running head using the {\tt fancyhdr} package which is included in the ICML 2014 style file package.  In case that the original title exceeds the size restrictions, a shorter form can be supplied by using

%\verb|\icmltitlerunning{...}|

%just before $\mathtt{\backslash begin\{document\}}$. Authors using \textbf{Word} must edit the header of the document themselves.

%\section{Format of the Paper} 
 
%All submissions must follow the same format to ensure the printer can reproduce them without problems and to let readers more easily find the information that they desire.

%\subsection{Length and Dimensions}

%Papers must not exceed eight (8) pages, including all figures, tables, and appendices, but excluding references. When references are included, the paper must not exceed nine (9) pages. Any submission that exceeds this page limit or that diverges significantly from the format specified herein will be rejected without review.

%The text of the paper should be formatted in two columns, with an overall width of 6.75 inches, height of 9.0 inches, and 0.25 inches between the columns. The left margin should be 0.75 inches and the top margin 1.0 inch (2.54~cm). The right and bottom margins will depend on whether you print on US letter or A4 paper, but all final versions must be produced for US letter size.

%The paper body should be set in 10~point type with a vertical spacing of 11~points. Please use Times  typeface throughout the text.
%Please use the default typeface (Computer Modern) throughout the text.

%\subsection{Title}

%The paper title should be set in 14~point bold type and centered between two horizontal rules that are 1~point thick, with 1.0~inch between the top rule and the top edge of the page. Capitalize the first letter of content words and put the rest of the title in lower case.

%\subsection{Author Information for Submission}
%\label{author info}

%To facilitate blind review, author information must not appear. If you are using \LaTeX\/ and the \texttt{icml2014.sty} file, you may use \verb+\icmlauthor{...}+ to specify authors.  The author information will simply not be printed until {\tt accepted} is an argument to the style file. Submissions that include the author information will not be reviewed.

%\subsubsection{Self-Citations}

%If your are citing published papers for which you are an author, refer to yourself in the third person. In particular, do not use phrases that reveal your identity (e.g., ``in previous work \cite{langley00}, we have shown \ldots'').

%Do not anonymize citations in the reference section by removing or blacking out author names. The only exception are manuscripts that are not yet published (e.g. under submission). If you choose to refer to such unpublished manuscripts \cite{anonymous}, anonymized copies have to be submitted as Supplementary Material via CMT. However, keep in mind that an ICML paper should be self contained and should contain sufficient detail for the reviewers to evaluate the work. In particular, reviewers are not required to look a the Supplementary Material when writing their review.

%\subsubsection{Camera-Ready Author Information}
%\label{final author}

%If a paper is accepted, a final camera-ready copy must be prepared.
%
%For camera-ready papers, author information should start 0.3~inches below the bottom rule surrounding the title. The authors' names should appear in 10~point bold type, electronic mail addresses in 10~point small capitals, and physical addresses in ordinary 10~point type. Each author's name should be flush left, whereas the email address should be flush right on the same line. The author's physical address should appear flush left on the ensuing line, on a single line if possible. If successive authors have the same affiliation, then give their physical address only once.

%A sample file (in PDF) with author names is included in the ICML2014 style file package.

%\subsection{Abstract}

%The paper abstract should begin in the left column, 0.4~inches below the final address. The heading `Abstract' should be centered, bold, and in 11~point type. The abstract body should use 10~point type, with a vertical spacing of 11~points, and should be indented 0.25~inches more than normal on left-hand and right-hand margins. Insert 0.4~inches of blank space after the body. Keep your abstract brief and  self-contained, limiting it to one paragraph and no more than six or seven sentences.

%\subsection{Partitioning the Text} 

%You should organize your paper into sections and paragraphs to help readers place a structure on the material and understand its contributions.

%\subsubsection{Sections and Subsections}

%Section headings should be numbered, flush left, and set in 11~pt bold type with the content words capitalized. Leave 0.25~inches of space before the heading and 0.15~inches after the heading.

%Similarly, subsection headings should be numbered, flush left, and set in 10~pt bold type with the content words capitalized. Leave 0.2~inches of space before the heading and 0.13~inches afterward.

%Finally, subsubsection headings should be numbered, flush left, and set in 10~pt small caps with the content words capitalized. Leave 0.18~inches of space before the heading and 0.1~inches after the heading. 

%Please use no more than three levels of headings.

%\subsubsection{Paragraphs and Footnotes}

%Within each section or subsection, you should further partition the paper into paragraphs. Do not indent the first line of a given paragraph, but insert a blank line between succeeding ones.
 
%You can use footnotes\footnote{For the sake of readability, footnotes should be complete sentences.} to provide readers with additional information about a topic without interrupting the flow of the paper. Indicate footnotes with a number in the text where the point is most relevant. Place the footnote in 9~point type at the bottom of the column in which it appears. Precede the first footnote in a column with a horizontal rule of 0.8~inches.\footnote{Multiple footnotes can appear in each column, in the same order as they appear in the text, but spread them across columns and pages if possible.}

%\begin{figure}[ht]
%\vskip 0.2in
%\begin{center}
%\centerline{\includegraphics[width=\columnwidth]{icml_numpapers}}
%\caption{Historical locations and number of accepted papers for International
%  Machine Learning Conferences (ICML 1993 -- ICML 2008) and
%  International Workshops on Machine Learning (ML 1988 -- ML
%  1992). At the time this figure was produced, the number of
%  accepted papers for ICML 2008 was unknown and instead estimated.}
%\label{icml-historical}
%\end{center}
%\vskip -0.2in
%\end{figure} 

%\subsection{Figures}
 
%You may want to include figures in the paper to help readers visualize your approach and your results. Such artwork should be centered, legible, and separated from the text. Lines should be dark and at least 0.5~points thick for purposes of reproduction, and text should not appear on a gray background.

%Label all distinct components of each figure. If the figure takes the form of a graph, then give a name for each axis and include a legend that briefly describes each curve. Do not include a title inside the figure; instead, the caption should serve this function.

%Number figures sequentially, placing the figure number and caption {\it after\/} the graphics, with at least 0.1~inches of space before the caption and 0.1~inches after it, as in Figure~\ref{icml-historical}.  The figure caption should be set in 9~point type and centered unless it runs two or more lines, in which case it should be flush left.  You may float figures to the top or bottom of a column, and you may set wide figures across both columns (use the environment {\tt figure*} in \LaTeX), but always place two-column figures at the top or bottom of the page.

%\subsection{Algorithms}

%If you are using \LaTeX, please use the ``algorithm'' and ``algorithmic'' environments to format pseudocode. These require the corresponding stylefiles, algorithm.sty and algorithmic.sty, which are supplied with this package. Algorithm~\ref{alg:example} shows an example. 

%\begin{algorithm}[tb]
%   \caption{Bubble Sort}
%   \label{alg:example}
%\begin{algorithmic}
%   \STATE {\bfseries Input:} data $x_i$, size $m$
%   \REPEAT
%   \STATE Initialize $noChange = true$.
%   \FOR{$i=1$ {\bfseries to} $m-1$}
%   \IF{$x_i > x_{i+1}$} 
%   \STATE Swap $x_i$ and $x_{i+1}$
%   \STATE $noChange = false$
%   \ENDIF
%   \ENDFOR
%   \UNTIL{$noChange$ is $true$}
%\end{algorithmic}
%\end{algorithm}
 
%\subsection{Tables} 
 
%You may also want to include tables that summarize material. Like figures, these should be centered, legible, and numbered consecutively. However, place the title {\it above\/} the table with at least 0.1~inches of space before the title and the same after it, as in Table~\ref{sample-table}. The table title should be set in 9~point type and centered unless it runs two or more lines, in which case it should be flush left.

% Note use of \abovespace and \belowspace to get reasonable spacing 
% above and below tabular lines. 

%\begin{table}[t]
%\caption{Classification accuracies for naive Bayes and flexible 
%Bayes on various data sets.}
%\label{sample-table}
%\vskip 0.15in
%\begin{center}
%\begin{small}
%\begin{sc}
%\begin{tabular}{lcccr}
%\hline
%\abovespace\belowspace
%Data set & Naive & Flexible & Better? \\
%\hline
%\abovespace
%Breast    & 95.9$\pm$ 0.2& 96.7$\pm$ 0.2& $\surd$ \\
%Cleveland & 83.3$\pm$ 0.6& 80.0$\pm$ 0.6& $\times$\\
%Glass2    & 61.9$\pm$ 1.4& 83.8$\pm$ 0.7& $\surd$ \\
%Credit    & 74.8$\pm$ 0.5& 78.3$\pm$ 0.6&         \\
%Horse     & 73.3$\pm$ 0.9& 69.7$\pm$ 1.0& $\times$\\
%Meta      & 67.1$\pm$ 0.6& 76.5$\pm$ 0.5& $\surd$ \\
%Pima      & 75.1$\pm$ 0.6& 73.9$\pm$ 0.5&         \\
%\belowspace
%Vehicle   & 44.9$\pm$ 0.6& 61.5$\pm$ 0.4& $\surd$ \\
%\hline
%\end{tabular}
%\end{sc}
%\end{small}
%\end{center}
%\vskip -0.1in
%\end{table}

%Tables contain textual material that can be typeset, as contrasted with figures, which contain graphical material that must be drawn. Specify the contents of each row and column in the table's topmost row. Again, you may float tables to a column's top or bottom, and set wide tables across both columns, but place two-column tables at the top or bottom of the page.
 
%\subsection{Citations and References} 

%Please use APA reference format regardless of your formatter or word processor. If you rely on the \LaTeX\/ bibliographic facility, use {\tt natbib.sty} and {\tt icml2014.bst} included in the style-file package to obtain this format.

%Citations within the text should include the authors' last names and year. If the authors' names are included in the sentence, place only the year in parentheses, for example when referencing Arthur Samuel's pioneering work \yrcite{Samuel59}. Otherwise place the entire reference in parentheses with the authors and year separated by a comma \cite{Samuel59}. List multiple references separated by semicolons \cite{kearns89,Samuel59,mitchell80}. Use the `et~al.' construct only for citations with three or more authors or after listing all authors to a publication in an earlier reference \cite{MachineLearningI}.

%Authors should cite their own work in the third person in the initial version of their paper submitted for blind review. Please refer to Section~\ref{author info} for detailed instructions on how to cite your own papers.

%Use an unnumbered first-level section heading for the references, and use a hanging indent style, with the first line of the reference flush against the left margin and subsequent lines indented by 10 points. The references at the end of this document give examples for journal articles \cite{Samuel59}, conference publications \cite{langley00}, book chapters \cite{Newell81}, books \cite{DudaHart2nd}, edited volumes \cite{MachineLearningI}, technical reports \cite{mitchell80}, and dissertations \cite{kearns89}. 

%Alphabetize references by the surnames of the first authors, with single author entries preceding multiple author entries. Order references for the same authors by year of publication, with the earliest first. Make sure that each reference includes all relevant information (e.g., page numbers).

%\subsection{Software and Data}

%We strongly encourage the publication of software and data with the camera-ready version of the paper whenever appropriate. This can be done by including a URL in the camera-ready copy. However, do not include URLs that reveal your institution or identity in your submission for review. Instead, provide an anonymous URL or upload the material as ``Supplementary Material'' into the CMT reviewing system. Note that reviewers are not required to look a this material when writing their review.


% Acknowledgements should only appear in the accepted version. 
%\section*{Acknowledgments} 
 
%\textbf{Do not} include acknowledgements in the initial version of the paper submitted for blind review.

%If a paper is accepted, the final camera-ready version can (and probably should) include acknowledgements. In this case, please place such acknowledgements in an unnumbered section at the end of the paper. Typically, this will include thanks to reviewers who gave useful comments, to colleagues who contributed to the ideas, and to funding agencies and corporate sponsors that provided financial support.  


% In the unusual situation where you want a paper to appear in the
% references without citing it in the main text, use \nocite
%\nocite{langley00}

\bibliography{biblio}
\bibliographystyle{icml2014}

\end{document} 


% This document was modified from the file originally made available by
% Pat Langley and Andrea Danyluk for ICML-2K. This version was
% created by Lise Getoor and Tobias Scheffer, it was slightly modified  
% from the 2010 version by Thorsten Joachims & Johannes Fuernkranz, 
% slightly modified from the 2009 version by Kiri Wagstaff and 
% Sam Roweis's 2008 version, which is slightly modified from 
% Prasad Tadepalli's 2007 version which is a lightly 
% changed version of the previous year's version by Andrew Moore, 
% which was in turn edited from those of Kristian Kersting and 
% Codrina Lauth. Alex Smola contributed to the algorithmic style files.  
